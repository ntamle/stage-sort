%%%%%%%%%%%%%%%%% DO NOT CHANGE HERE %%%%%%%%%%%%%%%%%%%% 
%%%%%%%%%%%%%%%%%%%%%%%%%%%%%%%%%%%%%%%%%%%%%%%%%%%%%%%%%%{
    \documentclass[twoside,11pt]{article}
    %%%%% PACKAGES %%%%%%
    \usepackage{pgm2016}
    \usepackage{amsmath}
    \usepackage{algorithm}
    \usepackage[noend]{algpseudocode}
    \usepackage{subcaption}
    \usepackage[english]{babel}	
    \usepackage{paralist}	
    \usepackage[lowtilde]{url}
    \usepackage{fixltx2e}
    \usepackage{listings}
    \usepackage{color}
    \usepackage{hyperref}
    \usepackage{stmaryrd}
    \usepackage{auto-pst-pdf}
    \usepackage{pst-all}
    \usepackage{pstricks-add}
    
    %%%%% MACROS %%%%%%
    \algrenewcommand\Return{\State \algorithmicreturn{} }
    \algnewcommand{\LineComment}[1]{\State \(\triangleright\) #1}
    \renewcommand{\thesubfigure}{\roman{subfigure}}
    \definecolor{codegreen}{rgb}{0,0.6,0}
    \definecolor{codegray}{rgb}{0.5,0.5,0.5}
    \definecolor{codepurple}{rgb}{0.58,0,0.82}
    \definecolor{backcolour}{rgb}{0.95,0.95,0.92}
    \lstdefinestyle{mystyle}{
       backgroundcolor=\color{backcolour},  
       commentstyle=\color{codegreen},
       keywordstyle=\color{magenta},
       numberstyle=\tiny\color{codegray},
       stringstyle=\color{codepurple},
       basicstyle=\footnotesize,
       breakatwhitespace=false,        
       breaklines=true,                
       captionpos=b,                    
       keepspaces=true,                
       numbers=left,                    
       numbersep=5pt,                  
       showspaces=false,                
       showstringspaces=false,
       showtabs=false,                  
       tabsize=2
    }
    \lstset{style=mystyle}
    \newcommand*{\SET}[1]  {\ensuremath{\mathbb{#1}}}
    \newcommand{\C}{\SET{C}}
    \newcommand{\R}{\SET{R}}
    \newcommand{\Z}{\SET{Z}}
    \newcommand{\N}{\SET{N}}
    \newcommand{\K}{\SET{K}}
    \DeclareMathOperator{\di}{d\!}
    \newcommand{\K}{\SET{K}}

    \begin{document}
    
    \title{Internship \assignmentNumber}
    
    \author{\name \studentName \email \studentEmail \\
    \studentNumber
    \addr
    }
    
    \maketitle
%%%%%%%%%%%%%%%%%%%%%%%%%%%%%%%%%%%%%%%%%%%%%%%%%%%%%%%%%%
%%%%%%%%%%%%%%%%%%%%%%%%%%%%%%%%%%%%%%%%%%%%%%%%%%%%%%%%%% }


\section{The sorting operation}

We consider the space $\mathbb{R}^n$. We can define the sorting operation as the function that output the sorted vector of a tuple $x \in \mathbb{R}^n$:
\begin{equation}
    sort : (x_1, ..., x_n) \longmapsto (x_{\sigma_1}, ..., x_{\sigma_n}) 
\end{equation}
where $\sigma \in \mathfrak{S}_n$ is any permutation such that $x_{\sigma_1} \leq  x_{\sigma_2} \leq ... \leq x_{\sigma_n}$. Let's note that $\sigma$ is not unique because we can have equality cases. However, the function is well-defined as we don't care how the equal elements are sorted.
% preciser
\begin{proposition}
The sorting operation is a semialgebraic function. 
\end{proposition}

\paragraph{Proof:} 
Let's recall that a semialgebraic set $A \subset \mathbb{R}^p$ is a finite union of sets of the form:
\begin{equation}
    \bigcap_{i=1}^k\: \{x \in \mathbb{R}^p \ | \ g_i(x) < 0, \ h_i(x) = 0 \}
\end{equation}\\
where the $g_i$ and $h_i$ are polynomial functions. The graph of a function $f : X \longrightarrow Y$ is the set:
\begin{equation}
    \Gamma_f = \{(x, y) \in X \times Y \ | \ y = f(x)  \}
\end{equation}
\\
One can write the graph of $sort$ as a finite union of affine (thus polynomial) constrained sets:
\begin{align*}
    \Gamma_{sort} &= \{(x, x_{\sigma}) \in (\mathbb{R}^n)^2 \ | \ \sigma \in \mathfrak{S}_n,\ x_{\sigma_1} \leq  x_{\sigma_2} \leq ... \leq x_{\sigma_n} \}
\end{align*}
% no big deal
Let's note that the fact inequalities are not strict is no big deal since we can rewrite it as a finite union of equalities and strict inequalities. The graph is then:

\begin{align*}
    \Gamma_{sort} = \bigcup_{ \sigma \in \mathfrak{S}_n} \bigcup_{I \subset \llbracket 1, n-1 \rrbracket} \{ (x, x_\sigma) \in (\mathbb{R}^n)^2 \ | \ &\forall i \in I,\: x_{\sigma_i} - x_{\sigma_{i+1}} < 0,\\ &\forall i \notin I\cup\{n\}, x_{\sigma_i} - x_{\sigma_{i+1}} = 0 \} 
\end{align*}
\\
which define a semialgebraic set.

\section{Conservative field for the sorting operation}
Here we try to define a conservative mapping for the function $sort$. Indeed, we can see that the function is differentiable almost everywhere, on the complement of an union of manifolds having a dimension strictly inferior to $p$. Precisely, given a permutation $\sigma \in \permut_p$, the Jacobian of $sort$ on the region $U_\sigma := \left\{x \in \mathbb{R}^p \ | \ x_{\sigma_1} < x_{\sigma_2} < ... < x_{\sigma_{p-1}} < x_{\sigma_p}  \right\}$ is the permutation matrix $P_\sigma$.
\\


\begin{proposition}[Conservative mapping for $sort$] Let $J : \R^p \rightrightarrows \R^{p \times p}$ be the set-valued map defined by:
\begin{equation}
\forall x \in \R^p,\  J(x) = \operatorname{conv}  \left\{P_\sigma \ | \ x_{\sigma_1} \leq x_{\sigma_2} \leq ... \leq x_{\sigma_p}\right\} 
\end{equation}
\end{proposition}
\noindent
Then $J$ is a conservative mapping for $sort$.
\end{document}
